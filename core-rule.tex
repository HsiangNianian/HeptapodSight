% Options for packages loaded elsewhere
\PassOptionsToPackage{unicode}{hyperref}
\PassOptionsToPackage{hyphens}{url}
\documentclass[
]{article}
\usepackage{xcolor}
\usepackage{amsmath,amssymb}
\setcounter{secnumdepth}{-\maxdimen} % remove section numbering
\usepackage{iftex}
\ifPDFTeX
  \usepackage[T1]{fontenc}
  \usepackage[utf8]{inputenc}
  \usepackage{textcomp} % provide euro and other symbols
\else % if luatex or xetex
  \usepackage{unicode-math} % this also loads fontspec
  \defaultfontfeatures{Scale=MatchLowercase}
  \defaultfontfeatures[\rmfamily]{Ligatures=TeX,Scale=1}
\fi
\usepackage{lmodern}
\ifPDFTeX\else
  % xetex/luatex font selection
\fi
% Use upquote if available, for straight quotes in verbatim environments
\IfFileExists{upquote.sty}{\usepackage{upquote}}{}
\IfFileExists{microtype.sty}{% use microtype if available
  \usepackage[]{microtype}
  \UseMicrotypeSet[protrusion]{basicmath} % disable protrusion for tt fonts
}{}
\makeatletter
\@ifundefined{KOMAClassName}{% if non-KOMA class
  \IfFileExists{parskip.sty}{%
    \usepackage{parskip}
  }{% else
    \setlength{\parindent}{0pt}
    \setlength{\parskip}{6pt plus 2pt minus 1pt}}
}{% if KOMA class
  \KOMAoptions{parskip=half}}
\makeatother
\usepackage{color}
\usepackage{fancyvrb}
\newcommand{\VerbBar}{|}
\newcommand{\VERB}{\Verb[commandchars=\\\{\}]}
\DefineVerbatimEnvironment{Highlighting}{Verbatim}{commandchars=\\\{\}}
% Add ',fontsize=\small' for more characters per line
\newenvironment{Shaded}{}{}
\newcommand{\AlertTok}[1]{\textcolor[rgb]{1.00,0.00,0.00}{\textbf{#1}}}
\newcommand{\AnnotationTok}[1]{\textcolor[rgb]{0.38,0.63,0.69}{\textbf{\textit{#1}}}}
\newcommand{\AttributeTok}[1]{\textcolor[rgb]{0.49,0.56,0.16}{#1}}
\newcommand{\BaseNTok}[1]{\textcolor[rgb]{0.25,0.63,0.44}{#1}}
\newcommand{\BuiltInTok}[1]{\textcolor[rgb]{0.00,0.50,0.00}{#1}}
\newcommand{\CharTok}[1]{\textcolor[rgb]{0.25,0.44,0.63}{#1}}
\newcommand{\CommentTok}[1]{\textcolor[rgb]{0.38,0.63,0.69}{\textit{#1}}}
\newcommand{\CommentVarTok}[1]{\textcolor[rgb]{0.38,0.63,0.69}{\textbf{\textit{#1}}}}
\newcommand{\ConstantTok}[1]{\textcolor[rgb]{0.53,0.00,0.00}{#1}}
\newcommand{\ControlFlowTok}[1]{\textcolor[rgb]{0.00,0.44,0.13}{\textbf{#1}}}
\newcommand{\DataTypeTok}[1]{\textcolor[rgb]{0.56,0.13,0.00}{#1}}
\newcommand{\DecValTok}[1]{\textcolor[rgb]{0.25,0.63,0.44}{#1}}
\newcommand{\DocumentationTok}[1]{\textcolor[rgb]{0.73,0.13,0.13}{\textit{#1}}}
\newcommand{\ErrorTok}[1]{\textcolor[rgb]{1.00,0.00,0.00}{\textbf{#1}}}
\newcommand{\ExtensionTok}[1]{#1}
\newcommand{\FloatTok}[1]{\textcolor[rgb]{0.25,0.63,0.44}{#1}}
\newcommand{\FunctionTok}[1]{\textcolor[rgb]{0.02,0.16,0.49}{#1}}
\newcommand{\ImportTok}[1]{\textcolor[rgb]{0.00,0.50,0.00}{\textbf{#1}}}
\newcommand{\InformationTok}[1]{\textcolor[rgb]{0.38,0.63,0.69}{\textbf{\textit{#1}}}}
\newcommand{\KeywordTok}[1]{\textcolor[rgb]{0.00,0.44,0.13}{\textbf{#1}}}
\newcommand{\NormalTok}[1]{#1}
\newcommand{\OperatorTok}[1]{\textcolor[rgb]{0.40,0.40,0.40}{#1}}
\newcommand{\OtherTok}[1]{\textcolor[rgb]{0.00,0.44,0.13}{#1}}
\newcommand{\PreprocessorTok}[1]{\textcolor[rgb]{0.74,0.48,0.00}{#1}}
\newcommand{\RegionMarkerTok}[1]{#1}
\newcommand{\SpecialCharTok}[1]{\textcolor[rgb]{0.25,0.44,0.63}{#1}}
\newcommand{\SpecialStringTok}[1]{\textcolor[rgb]{0.73,0.40,0.53}{#1}}
\newcommand{\StringTok}[1]{\textcolor[rgb]{0.25,0.44,0.63}{#1}}
\newcommand{\VariableTok}[1]{\textcolor[rgb]{0.10,0.09,0.49}{#1}}
\newcommand{\VerbatimStringTok}[1]{\textcolor[rgb]{0.25,0.44,0.63}{#1}}
\newcommand{\WarningTok}[1]{\textcolor[rgb]{0.38,0.63,0.69}{\textbf{\textit{#1}}}}
\usepackage{longtable,booktabs,array}
\newcounter{none} % for unnumbered tables
\usepackage{calc} % for calculating minipage widths
% Correct order of tables after \paragraph or \subparagraph
\usepackage{etoolbox}
\makeatletter
\patchcmd\longtable{\par}{\if@noskipsec\mbox{}\fi\par}{}{}
\makeatother
% Allow footnotes in longtable head/foot
\IfFileExists{footnotehyper.sty}{\usepackage{footnotehyper}}{\usepackage{footnote}}
\makesavenoteenv{longtable}
\setlength{\emergencystretch}{3em} % prevent overfull lines
\providecommand{\tightlist}{%
  \setlength{\itemsep}{0pt}\setlength{\parskip}{0pt}}
\usepackage{bookmark}
\IfFileExists{xurl.sty}{\usepackage{xurl}}{} % add URL line breaks if available
\urlstyle{same}
\hypersetup{
  hidelinks,
  pdfcreator={LaTeX via pandoc}}

\author{}
\date{}

\begin{document}

\section{\texorpdfstring{《七肢桶之视》
}{《七肢桶之视》 }}\label{ux4e03ux80a2ux6876ux4e4bux89c6}

\subsection{\texorpdfstring{\emph{Heptapod Sight}
}{Heptapod Sight }}\label{heptapod-sight}

\begin{quote}
A Tabletop Role-Playing Game of Foreknown Love
\end{quote}

\emph{``目的与方法,本是一体。''} ------ 灵感源自 Ted
Chiang《你一生的故事》

\textbf{Warning}\\
\textbf{免责声明}\\
本项目为非官方同人创作,与作者 Ted
Chiang、其代理人或任何版权持有方无任何 affiliation。\\
规则文本采用
\textbf{\href{https://creativecommons.org/licenses/by-sa/4.0/}{CC BY-SA
4.0}} 许可(见文末)。\\
你可自由使用、改编、分发------只需署名并以相同方式共享。

\begin{center}\rule{0.5\linewidth}{0.5pt}\end{center}

\subsection{目录}\label{ux76eeux5f55}

\begin{itemize}
\item
  \hyperref[-ux5f15ux8a00ux8fd9ux4e0dux662fux6e38ux620fux800cux662fux4e00ux573aux7ea6ux5b9a]{引言:这不是游戏,而是一场约定}
\item
  \hyperref[-ux89d2ux8272ux521bux5efaux7ec7ux547dux4e09ux95ee]{角色创建:织命三问}
\item
  \hyperref[-ux6838ux5fc3ux673aux5236ux53ccux8f68ux884cux8fdb]{核心机制:双轨行进}
\item
  \hyperref[-ux8bedux8a00ux5373ux4eeaux5f0fux521dux8bedux7cfbux7edf]{语言即仪式:初语系统}
\item
  \hyperref[-ux5f15ux5bfcux8005ux4e4bux827a]{引导者之艺}
\item
  \hyperref[-ux9644ux5f55ux5febux901fux53c2ux8003ux5361]{附录:快速参考卡}
\item
  \hyperref[-ux8bb8ux53efux4e0eux81f4ux8c22]{许可与致谢}
\end{itemize}

\begin{center}\rule{0.5\linewidth}{0.5pt}\end{center}

\subsection{▍
引言:这不是游戏,而是一场约定}\label{ux258d-ux5f15ux8a00ux8fd9ux4e0dux662fux6e38ux620fux800cux662fux4e00ux573aux7ea6ux5b9a}

\begin{quote}
\emph{``学习七肢桶语言,不是为了预测未来------而是为了,在看见终点后,依然能迈出第一步。''}
\end{quote}

《七肢桶之视》是一套\textbf{实验性桌面角色扮演游戏(TRPG)规则},探索一个悖论:\\
\textbf{当你已看见自己的一生------包括所有欢愉与痛楚------你是否仍愿开始?}

这不是关于``改变命运'',而是关于\textbf{拥抱命运};\\
不是模拟世界如何运作,而是体验\textbf{人如何与已知的结局共处}。

\subsubsection{✦ 适合谁?}\label{ux2726-ux9002ux5408ux8c01}

\begin{itemize}
\item
  想尝试\textbf{无战斗、高情感密度}TRPG 的玩家
\item
  文学/哲学爱好者,尤爱特德·姜、村上春树、是枝裕和
\item
  2--4 人小团体(1 名引导者 + 1--3 名叙述者)
\item
  60--120 分钟的短团(One-shot)或系列微团(Mini-Campaign)
\end{itemize}

\subsubsection{✦ 你需要}\label{ux2726-ux4f60ux9700ux8981}

{\def\LTcaptype{none} % do not increment counter
\begin{longtable}[]{@{}ll@{}}
\toprule\noalign{}
物品 & 说明 \\
\midrule\noalign{}
\endhead
\bottomrule\noalign{}
\endlastfoot
纸与笔 & 每人至少 1 张;\textbf{手写是仪式的一部分} \\
3 枚 d6 & 或使用 \href{https://roll20.net/}{Roll20} / Discord
\texttt{/roll\ 3d6} \\
一个安静空间 & 光线柔和,无打扰;建议开场前静默 1 分钟 \\
一颗心 & 明知结局,仍愿赴约 \\
\end{longtable}
}

\textbf{Note}\\
\textbf{术语说明}

\begin{itemize}
\item
  \textbf{叙述者(Narrator)}:玩家;非``扮演角色'',而是``重述一段已知的生命''
\item
  \textbf{引导者(Guide)}:原 GM;不掌控剧情,只协助记忆浮现
\item
  \textbf{环视力(Circular Sight)}:资源点;象征``接纳命运的勇气''
\end{itemize}

\begin{center}\rule{0.5\linewidth}{0.5pt}\end{center}

\subsection{▍
角色创建:织命三问}\label{ux258d-ux89d2ux8272ux521bux5efaux7ec7ux547dux4e09ux95ee}

角色不是``构建''出来的,而是\textbf{从未来回望时浮现的轮廓}。\\
请每位叙述者\textbf{手写}回答以下三问------答案将成为你角色的``时间锚点''。

\subsubsection{\texorpdfstring{1. 光核(Lumen)
}{1. 光核(Lumen) }}\label{1-ux5149ux6838lumen}

\begin{quote}
\emph{``你已看见自己生命中最明亮的一刻------它为何明亮?''}
\end{quote}

示例:

\begin{quote}
\emph{``她踮脚把野花插在我耳后,说`妈妈像春天'。''}\\
\emph{``化疗结束那天,他煮糊了粥,我们笑到咳嗽。''}
\end{quote}

\textbf{作用}:当角色濒临放弃,可引用光核重获 1 点环视力。

\begin{center}\rule{0.5\linewidth}{0.5pt}\end{center}

\subsubsection{\texorpdfstring{2. 重锚(Anchor)
}{2. 重锚(Anchor) }}\label{2-ux91cdux951aanchor}

\begin{quote}
\emph{``你已看见自己生命中最沉重的一刻------你仍选择走向它吗?''}
\end{quote}

示例:

\begin{quote}
\emph{``手术灯亮起前,她握着我的手问:`会疼吗?' 我说`不会'。''}\\
\emph{``离婚协议签字时,窗外玉兰正落,像一场慢镜头的雪。''}
\end{quote}

\textbf{规则}:

\begin{quote}
\begin{itemize}
\item
  此事件\textbf{必须}在团中发生(引导者确保其浮现)
\item
  回避它将导致「目的偏移」惩罚(见\hyperref[-ux6838ux5fc3ux673aux5236ux53ccux8f68ux884cux8fdb]{核心机制})
\end{itemize}
\end{quote}

\begin{center}\rule{0.5\linewidth}{0.5pt}\end{center}

\subsubsection{\texorpdfstring{3. 初语(First Word)
}{3. 初语(First Word) }}\label{3-ux521dux8bedfirst-word}

\begin{quote}
\emph{``若语言能重塑现实,你愿为所爱之人写下怎样的第一句话?''}
\end{quote}

要求:

\begin{quote}
\begin{itemize}
\item
  1 个完整句子
\item
  \textbf{必须以 ``我已看见\ldots\ldots'' 开头}
\item
  聚焦\textbf{自身感知与关系},非命令他人
\end{itemize}
\end{quote}

示例:

\begin{quote}
\emph{``我已看见你在雨中转身,伞沿滴落的水珠里,有我们未曾说出的原谅。''}\\
\emph{``我已看见十年后,你教孩子唱那首我哄你入睡的歌,调子全跑了。''}
\end{quote}

\textbf{作用}:见\hyperref[-ux8bedux8a00ux5373ux4eeaux5f0fux521dux8bedux7cfbux7edf]{初语系统}

\begin{center}\rule{0.5\linewidth}{0.5pt}\end{center}

\subsubsection{▶ 角色卡模板(建议 8×12 cm
卡片)}\label{ux25b6-ux89d2ux8272ux5361ux6a21ux677fux5efaux8bae-8uxd712-cm-ux5361ux7247}

\begin{Shaded}
\begin{Highlighting}[]
\NormalTok{[你的名字,或“母亲”/“医生”/“姐姐”……]}

\NormalTok{● 光核:……}
\NormalTok{◎ 重锚:……}
\NormalTok{◎ 初语:……}
\NormalTok{(环视力:●●●)}
\end{Highlighting}
\end{Shaded}

\begin{quote}
提示:角色名可用第二人称``你'',强化代入感------\\
\emph{你,守夜人;你,未寄出的信;你,融化的糖}。
\end{quote}

\begin{center}\rule{0.5\linewidth}{0.5pt}\end{center}

\subsection{▍
核心机制:双轨行进}\label{ux258d-ux6838ux5fc3ux673aux5236ux53ccux8f68ux884cux8fdb}

时间不是线,而是环。\\
每轮行动,你同步存在于两条轨道:

{\def\LTcaptype{none} % do not increment counter
\begin{longtable}[]{@{}
  >{\raggedright\arraybackslash}p{(\linewidth - 6\tabcolsep) * \real{0.2500}}
  >{\raggedright\arraybackslash}p{(\linewidth - 6\tabcolsep) * \real{0.2500}}
  >{\raggedright\arraybackslash}p{(\linewidth - 6\tabcolsep) * \real{0.2500}}
  >{\raggedright\arraybackslash}p{(\linewidth - 6\tabcolsep) * \real{0.2500}}@{}}
\toprule\noalign{}
轨道 & 名称 & 操作 & 示例 \\
\midrule\noalign{}
\endhead
\bottomrule\noalign{}
\endlastfoot
\vtop{\hbox{\strut \textbf{线性轨}}\hbox{\strut (The Path)}} &
``我正在经历\ldots\ldots'' & 描述当下言行(符合人类因果) &
\emph{``我伸手,想擦去她眼角的泪。''} \\
\vtop{\hbox{\strut \textbf{环形轨}}\hbox{\strut (The Circle)}} &
``我早已看见\ldots\ldots'' &
\vtop{\hbox{\strut 插入一段\textbf{已完成的未来记忆}}\hbox{\strut (必须用过去时;每轮限
1 次)}} &
\emph{``我记得那滴泪落在你手背,十年后我仍记得它的温度。''} \\
\end{longtable}
}

\begin{quote}
\textbf{设计意图}:环形轨不是``预知'',而是``记忆''------因对七肢桶而言,未来与过去同样真实。
\end{quote}

\begin{center}\rule{0.5\linewidth}{0.5pt}\end{center}

\subsubsection{▶ 目的骰(Telos
Dice)检定流程}\label{ux25b6-ux76eeux7684ux9ab0telos-diceux68c0ux5b9aux6d41ux7a0b}

\textbf{仅当选择有代价时检定}(非所有对话都需要!)。

\begin{enumerate}
\def\labelenumi{\arabic{enumi}.}
\item
  \textbf{宣告目的与方法}

  \begin{quote}
  \emph{``我的目的是 \textbf{\_\_},因此我此刻选择 \textbf{\_\_}。''}

  目的 ≠ 情绪(如``让她开心'')\\
  目的 = \textbf{终极意义}(如``让她知道爱不必完美'')

  方法 ≠ 动作(如``说`我爱你'\,'')\\
  方法 = \textbf{具体姿态}(如``把草莓糖放回她口袋,不说破'')
  \end{quote}
\item
  \textbf{掷 3d6},查表:
\end{enumerate}

{\def\LTcaptype{none} % do not increment counter
\begin{longtable}[]{@{}
  >{\raggedright\arraybackslash}p{(\linewidth - 4\tabcolsep) * \real{0.3333}}
  >{\raggedright\arraybackslash}p{(\linewidth - 4\tabcolsep) * \real{0.3333}}
  >{\raggedright\arraybackslash}p{(\linewidth - 4\tabcolsep) * \real{0.3333}}@{}}
\toprule\noalign{}
总和 & 结果 & 引导者应做 \\
\midrule\noalign{}
\endhead
\bottomrule\noalign{}
\endlastfoot
\textbf{3--8} & \vtop{\hbox{\strut \textbf{环闭合}}\hbox{\strut (Circle
Closes)}} &
行动成功;\textbf{叙述者必须立即补全一条新记忆}(过去时),并 +1
环视力 \\
\textbf{9--12} &
\vtop{\hbox{\strut \textbf{路径延续}}\hbox{\strut (Path Continues)}} &
\vtop{\hbox{\strut 行动部分成功;\textbf{引导者提议 1
个代价}:}\hbox{\strut • 失去 1 环视力}\hbox{\strut •
重锚提前显现征兆}\hbox{\strut • 光核暂时模糊(本轮不可用)}} \\
\textbf{13--18} &
\vtop{\hbox{\strut \textbf{目的偏移}}\hbox{\strut (Telos Drifts)}} &
行动失败;\textbf{引导者揭示一个未预见的未来片段}(30
秒内口述),常含苦涩真相 \\
\end{longtable}
}

\begin{quote}
\textbf{关键原则}

\begin{itemize}
\item
  高骰 ≠ 好事:18
  可能意味着``你终于说出`我爱你'------却在她已听不见的病床前''
\item
  检定不问``能否做到'',而问``\textbf{是否愿承担其意义}''
\end{itemize}
\end{quote}

\begin{center}\rule{0.5\linewidth}{0.5pt}\end{center}

\subsection{▍
语言即仪式:初语系统}\label{ux258d-ux8bedux8a00ux5373ux4eeaux5f0fux521dux8bedux7cfbux7edf}

语言不是工具,而是\textbf{重塑现实的仪式}。

\subsubsection{✦ 环视力(Circular
Sight)}\label{ux2726-ux73afux89c6ux529bcircular-sight}

\begin{itemize}
\item
  起始:每人 \textbf{3 点}(●●●)
\item
  \textbf{获取}:

  \begin{itemize}
  \item
    主动接纳苦涩记忆(不辩解/转移),+1
  \item
    用初语促成他人「环闭合」,+1
  \end{itemize}
\item
  \textbf{消耗}:见下表
\end{itemize}

{\def\LTcaptype{none} % do not increment counter
\begin{longtable}[]{@{}
  >{\raggedright\arraybackslash}p{(\linewidth - 4\tabcolsep) * \real{0.3333}}
  >{\raggedright\arraybackslash}p{(\linewidth - 4\tabcolsep) * \real{0.3333}}
  >{\raggedright\arraybackslash}p{(\linewidth - 4\tabcolsep) * \real{0.3333}}@{}}
\toprule\noalign{}
行动 & 消耗 & 效果 \\
\midrule\noalign{}
\endhead
\bottomrule\noalign{}
\endlastfoot
\textbf{启动初语} & 1 点 &
\vtop{\hbox{\strut 写下你的初语句(必须手写/打字),依内容触发:}\hbox{\strut •
描述\textbf{未发生}之事 → 成为未来记忆库条目}\hbox{\strut •
描述\textbf{正发生}之事 → 重掷 1 枚骰}\hbox{\strut •
描述\textbf{已发生却未被承认}之事 → 全体静默 1 分钟,共同补全细节}} \\
\textbf{重写偏移} & 2 点 &
\vtop{\hbox{\strut 将一次「目的偏移」的未来片段,改写为:}\hbox{\strut \emph{``我早已选择如此。''}
→ 转为「环闭合」,并获得该记忆}} \\
\end{longtable}
}

\subsubsection{✦
初语使用守则}\label{ux2726-ux521dux8bedux4f7fux7528ux5b88ux5219}

\begin{itemize}
\item
  不可命令他人行为(\emph{``我已看见你原谅我''})
\item
  必须聚焦\textbf{自身感知与关系}(\emph{``我已看见你转身时,风掀起你衣角像一只飞走的鸟''})
\item
  一旦写下,\textbf{即成为世界事实}------引导者须将其编织进后续叙事
\end{itemize}

\begin{quote}
\textbf{场景示例}\\
\emph{情境:女儿坚持独自旅行,你欲阻拦。}\\
你说:``我的目的是让她知道家永远等她,因此我此刻选择把熊塞进她手心。''\\
掷骰:9 → 路径延续\\
引导者:``她收下熊,但低声说:`妈,别拍照。'\,''(代价:你忍住没举起手机)\\
你消耗 1 环视力,写下初语:\\
\emph{``我已看见十年后,她抱着这熊,教自己的孩子唱那首你哄她入睡的歌。''}\\
→ GM 描述:她背包侧袋露出熊的一只脚,毛已磨白。
\end{quote}

\begin{center}\rule{0.5\linewidth}{0.5pt}\end{center}

\subsection{▍ 引导者之艺}\label{ux258d-ux5f15ux5bfcux8005ux4e4bux827a}

你不是``掌控者'',而是\textbf{时间织工}(Weaver of Time)。

\subsubsection{✦
三节点结构(必用)}\label{ux2726-ux4e09ux8282ux70b9ux7ed3ux6784ux5fc5ux7528}

每场团围绕三个\textbf{时间锚点}展开:

{\def\LTcaptype{none} % do not increment counter
\begin{longtable}[]{@{}
  >{\raggedright\arraybackslash}p{(\linewidth - 4\tabcolsep) * \real{0.3333}}
  >{\raggedright\arraybackslash}p{(\linewidth - 4\tabcolsep) * \real{0.3333}}
  >{\raggedright\arraybackslash}p{(\linewidth - 4\tabcolsep) * \real{0.3333}}@{}}
\toprule\noalign{}
节点 & 引导者准备 & 叙述者体验 \\
\midrule\noalign{}
\endhead
\bottomrule\noalign{}
\endlastfoot
\textbf{初遇之日} & \vtop{\hbox{\strut 一个平凡场景 +
一个微小异常}\hbox{\strut (例:钟停了 / 茶凉得快 /
孩子问``时间有味道吗?'')}} & 在未知中行动,但环形轨已有模糊闪回 \\
\textbf{裂隙之时} & 一次未说出口的话 / 一个被误解的选择 &
重锚事件发生;光核首次受挑战 \\
\textbf{终语之刻} &
\vtop{\hbox{\strut \textbf{开场即告知所有人结局场景}}\hbox{\strut (例:``一年后,你在她空房间整理遗物,窗台上有半块融化的草莓糖'')}}
& 所有行动朝此汇聚;终局不是终点,而是理解的起点 \\
\end{longtable}
}

\begin{quote}
\textbf{开场仪式}:\\
全体闭眼 1 分钟,回想:\\
\emph{``你生命中,哪一刻让你觉得------即使知道结局,仍愿重来一遍?''}\\
睁眼后,每人轻声说一个词(不解释)。这些词将成为本场「世界基调」。
\end{quote}

\subsubsection{✦
当玩家问``未来会怎样?''}\label{ux2726-ux5f53ux73a9ux5bb6ux95eeux672aux6765ux4f1aux600eux6837}

正确回应:

\begin{quote}
\emph{``你已看见它------现在,你想如何走向它?''}\\
然后递纸笔:\emph{``写下你记忆中的一个细节。''}
\end{quote}

禁止:

\begin{itemize}
\item
  隐藏结局
\item
  用骰子逃避哲学重量(灌铅)
\end{itemize}

\subsubsection{✦
团结束仪式}\label{ux2726-ux56e2ux7ed3ux675fux4eeaux5f0f}

请所有人写下:

\begin{quote}
\emph{``若重来一次,我仍会选择 \textbf{\_\_}。''}\\
收集纸条,不读出------封存,或焚于静默。
\end{quote}

\begin{center}\rule{0.5\linewidth}{0.5pt}\end{center}

\subsection{▍
附录:快速参考卡}\label{ux258d-ux9644ux5f55ux5febux901fux53c2ux8003ux5361}

\begin{Shaded}
\begin{Highlighting}[]
\NormalTok{【掷骰流程】}
\NormalTok{1. 说:“我的目的是…,因此我选择…”}
\NormalTok{2. 掷 3d6 →}
\NormalTok{   3–8 ◎ 环闭合:成功 + 补记忆 +1环视力}
\NormalTok{   9–12 ● 路径延续:成功但付代价}
\NormalTok{   13–18 目的偏移:失败 + 揭示苦涩未来}

\NormalTok{【初语】}
\NormalTok{{-} 花1环视力,写“我已看见……”}
\NormalTok{{-} 未发生 → 成记忆}
\NormalTok{{-} 正发生 → 重掷1骰}
\NormalTok{{-} 已发生未认 → 静默1分钟}

\NormalTok{【环视力】}
\NormalTok{{-} 起始3}
\NormalTok{{-} 接纳痛苦 +1}
\NormalTok{{-} 重写偏移需2}
\end{Highlighting}
\end{Shaded}

\begin{center}\rule{0.5\linewidth}{0.5pt}\end{center}

\subsection{▍ 许可与致谢}\label{ux258d-ux8bb8ux53efux4e0eux81f4ux8c22}

\begin{itemize}
\item
  \textbf{文本许可}:\href{https://creativecommons.org/licenses/by-sa/4.0/}{CC
  BY-SA 4.0}\\
  → 你可自由分享、改编,只需:\\
  (1) 署名原作者(例:\emph{Based on ``Heptapod Sight'' by
  hsiangnianian})\\
  (2) 以相同许可发布衍生作品
\item
  \textbf{特别致谢}:

  \begin{itemize}
  \item
    Ted Chiang,《你一生的故事》------时间、语言与爱的永恒叩问
  \item
    Emily Care Boss, \emph{The White Box} ------ 实验性 RPG 设计的灯塔
  \item
    所有敢于在已知结局中仍选择开始的人
  \end{itemize}
\end{itemize}

\begin{quote}
\begin{quote}
愿你在环形时间中,找到线性行走的勇气。\\
------ hsiangnianian, 2025
\end{quote}
\end{quote}

\end{document}
